\documentclass{article}

\usepackage[margin = 1in]{geometry}
\usepackage{hyperref}
\hypersetup{colorlinks = TRUE, citecolor=blue}
\usepackage{amsmath}
\usepackage{amssymb}
\usepackage{graphicx}
\usepackage[export]{adjustbox}
\usepackage{natbib}
\linespread{2}
\usepackage{indentfirst}

\begin{document}

\section{?All models are wrong. Some models are useful. (George Box)}

\section{Introduction to Genomics}
\subsection{What is DNA?}
% first concepts to define: DNA, gene, expression
% what genes do

\subsection{Common Misconceptions}

DNA does not define everything about you.
- which genes are "turned on/off"
- different upbringings and environments


\subsection{(Define Jargon) Genomic Variation}

\subsection{Why study Genomic Variation?}
- Important for health and precision medicine
- Catalogs different ways in which people's genomes are healthy.
- When looking for disease-related genomic variants, we can use this information to rule out healthy variants.

\subsection{Natural Selection}

Thus, a variant that occurs naturally in a population likely doesn't cause disease symptoms.

\subsection{(Define Jargon) SNPs}
Single Nucleotide Polymorphism
poly- many
morph - 
Why SNP data ?

\subsection{Data (vcf file)}


\end{document}